\documentclass[a4paper, 11pt]{article}
\begin{document}
\section{Requirements}
\subsection{Transactions}
We have a set of Threads $T$ each representing a transaction.The code for a 
Transaction is like below:

\hspace{12mm}	Thread.doIt(new Callable)()\{

\hspace{12mm}		// code for transaction

\hspace{12mm}	\}

This is adopted from dstm and can and may be changed. Within this block which 
represents a member of set $T$. The functions beginTransction and endTransaction are implicitly called by the system without the intervention of the programmer.However, the programmer can read and write from TransactionalFile Objects within the blocks of a member of $T$. To provide consisteny the read and write from ordinary Java object files should be prohibited, otherwise the semantics of transactions would not be preserved. Hence, the sequence of operations provided to the programmer within a member of $T$ is the set \{TransactionalFile.Read(), TransactionalFile.Write()i, TransactionalFile.Seek()\}. However, always a beginTransaction() and commitTransaction() (if the transaction does not abort prior to this point) are performed too. \\ 

A Transaction consists of a sequence of operations. Thus, a transaction can be represented as the \{$op_1$, $op_2$, ..., $op_n$\}. The flow of program should be in a manner that we could have an arbitrary serial order of the members of $T$ (e.g. $T_5,T_6,T_1$,...) or any other order. The serial order should be in such a way that for any given two members of the set $T$, it looks all the operations of one occur before all the operation of the other. %In the first case we would say the $T_i$ precedes $T_j$ and in the seconds case $T_j$ preceds $T_i$ and $T_i$ can observe all the effects made by the operations in $T_j$. \\

\subsection{Defenitions}

Def 1- \emph{Set of Operations}: Operations are taken from the set \{readoffset(fildescriptor), getoffset(filedescriptor), writeoffset(filedescriptor), readdata(inode, offset, length), writedata(inode, offset, length), commit\}. We denote read operations as readoffset and readdata and write operations as writeoffset and writedata.\\

Def 2- \emph{Operations Sharing Resources} 1- A readoffset operation and writeoffset operations are said to be sharing resources if they both operate on the same filedescriptor. 2- A readdata and writedata operations are said to be sharing resources if they both operate on the same inode AND the range of (offset, offset + length) within one overlaps with that of the other.\\ 

Def 3- A read operation can only see the writes made by write operations sharing reources.\\ 

Def 4- \emph{Commit Instant and Commit Operation}: A transaction commits when it invokes the "commit" operation. A transaction $T_i$ is said to "commit" at instant $t_i$, if and only if it reflects its "writes" mabe by write operations in the filesystem or equivalently makes it writes visible to the whole system for all members of $T$ accesing the data at instant $t_j$ such that $t_j > t_i$. After a transaction invokes commit operation, the transactin ends and no more operation by transaction is done. An operation $\in OP_{T_i}$ is said to commit if and only if $T_i$ commits. Commit instant are not splittable, hence it appears to other transactions that all writes are refleted in the file system together.\\ 

Def 5- \emph{Precedence Relationship}: If $OP_{executed} = \{op_0, op_1, ..., op_n\}$ represnts the operations executed so far(from all transactions running), and the indices represnt the order of operations executed so far in an ascending manner, we define $op_i \rightarrow op_j$ (precedes) if and only if $i<j$.\\ 

Def 6- \emph{Visbility of Writes}: Assume $T_i$ is an uncommitted transaction and $T_{commited}$ indicates the set of commited transaction (those which have invoked the "commit" operation). A read operation $b \in OP_{T_i}$ MAY ONLY see the writes(changes) made by precedent write operation $a$ that shares resources with $b$ AND also is subject to one of the following conditions:\\

\hspace{7mm} 1- $a \in OP_{T_j}$ ($j \neq i$) such that $T_j \in T_{commited}$ \\

\hspace{7mm} 2- $a \in OP_{T_i}$ and $a$ happens before $b$ in the natural order of the transaction. Formally, $a \rightarrow b$\\
 

Corrolary 1- \emph{No See in The Fututre}: If $ \exists op_i \in WRITEOP_{T_i}$ and $\exists op_j \in READOP{T_j}$ such that $i \neq j$ and $op_i \rightarrow op_j$ then if $op_j \rightarrow commit-operation_{T_i}$, writes made by $op_j$ are not seen by $op_i$.i\\

proof: Follows immediately from Def 5 and Def 6.\\

Corrolary 2- \emph{Read Must See Most Recent Comitted Write}: If $a \in OP{T_i}$ reads the resource $r_i$ at $t_i$. In case $a$ has multiple writer precedessors sharing $r_i$ (denoted as $OP_{precedessors-for-r_i}$), then if all of them are from sets other than $OP_{T_i}$, $a$ sees the writes made by $op_{T_j} \in OP_{precedessors-for-r_i}$ such that $T_j$ has the greatest "commit instant" less than $t_i$ between all transaction with such operations . Otherwise, $a$ sees the most recent precedessor in $OP_{T_i}$. \\

axiom: Assume $a$ is accessing data at $t_i$, according to Def 6, only writes by those transactions that have already committed would be visble. Hence, $t_i$ is neccasarily gretaer than all members of \{$committime_{T_1}, ..., commitime_{T_{i-1}}$\},and hence the writes to $r_i$ made by these commited transaction, have accroding to Def 4 been already reflected in the file system. Since, $T_{i-1}$ has the greatest commit instant its changes have overriden that of previously commited transactions and thus, $a$ reading $r_i$ from file system sees these changes.\\

In case 2 that there are writer operations writng $r_i$ and preceding $a$ in $OP_{T_i}$, according to defenition of transaction, $r_i$ should be read from the writes made by write operations in $T_i$, and as the last of such write operation overrides the written data to $r_i$ by other operations, $a$ gets to see the most recent precedessor in $T_i$.\\   


Def 7- $\forall$ $op_{T_i}$ $\in OP_{T_i}$ and $\forall op_{T_j}\in OP_{T_j}$ if and only if $op_{T_i} \rightarrow op_{T_j}$ then $T_i$ $\rightarrow T_j$ \hspace{8mm} (this defines precedes relationship for members of $T$) \\

Def 8- \emph{Correctness}: A sequence of tarnsaction are said to be consistent if and only if a total ordeing of them according to precedence relationship can be established that demonstrates the same behavior as the execution of the program. Behavior for an operation means the data it has read. Demonstrating the same behavior thus means all the read operations should still see the same data in the new sequence as they have seen in the actual sequence.\\

Note: Def 7 and Def 8 indicate that if operations can be commuted in a given sequnce of $OP_{executed} = \{op_{T_1}(0), op_{T_4}(0), ... op_{T_1}(n)\}$ such that a total ordering of transactions (e.g \{$OP_{T_1}, OP_{T_2}, ..., OP_{T_n}) $\} can be obtained then the execution is consistent and correct. The eligiblity to commute is subject to conforming to Corrolary 1 \& 2.\\

Def 9- \emph{Relocation Operations in A Sequence}: In a sequence of operations $OP = \{op_1, op_2, ...,op_n\}$, any operation can relocate its position in the sequence unless as a result of this change, the behavior of a read operation changes (it reads different data).\\

Note: \emph{No Exchange for Operations Belonging to The Same Transaction}: If $op_i \in OP_{T_i}$ and $op_{i+1} \in OP_{T_i}$, we never exchange $op_i$ and $op_{i+1}$, since even if this exchange does not change any operations behavior, there is still no point in doing this as the aim is to put all operations belonging to the same transaction together, the internal order among these is not any of our concern, and the precedence relationship among these as indicated by the execution should be maintained.\\ 


%Def 9- A pair of operations are said to be conflictant if they form a pair of (read, write) or (write, read) acting on the same resource $r_n$.  

%Corrolary 3: \emph{For $T_i$ to Commit All Members of $T_{commit}$ Should Precede $T_i$}

%proof: Follows immediately from Def 7 and Def 8. This is later proven formally.\\

\subsection{Rules}

Rule 1- $\forall op_i \in  OP{T_i}$, $\forall op_j \in  OP{T_j}$, $\forall op_k \in  OP{T_k}$, if $op_i \rightarrow op_k$ and  $op_k \rightarrow op_j$, then $op_i \rightarrow op_j$ \\

proof: Follows from the defenition of $\rightarrow$ \\


Rule 2- If $T_i \rightarrow T_k$ and $T_k \rightarrow T_j$ then $T_i \rightarrow T_j$ \\

proof: Follws from Rule 1 and Def 7. \\ 

Rule 3- \emph{Exchnaging Position of Consecutive Operations}: Formally having the sequence of operations $OP = \{op_1,...,op_i,op_{i+1},...,op_n\}$, if $op_i \in OP_{T_n}$ and $op_{i+1} \in OP{T_m}$ such that $ n\neq m$, $op_i$ and $op_{i+1}$ can exchange positions if and only if none of the follwing conditions apply:\\

\hspace{8mm} 1- If $op_i = {commit}$ and $op_{i+1} = {read}$ and $op_{i+1}$ reads $r_k$, then if there is at least another operation in $T_n$ that writes the data $r_k$, $op_i$ and $op_{i+1}$ can not exchange positions.\\ 

\hspace{8mm} 2- If $op_{i+1} = {commit}$ and $op_{i} = {read}$ and $op_{i}$ reads $r_k$, then if there is at least another operation in $T_m$ that writes the data $r_k$, $op_i$ and $op_{i+1}$ can not exchange positions.\\ 


proof: According to Def 9, none of the operatios in the sequence should change behavior as the result of this exchange, however in this argument only $op_i$ and $op_{i+1}$ may change behavior as those are the only operations that their postions in the sequence is changed. However, since behavior is only defined for read operations, one of these without losing generality lets say $op_i$ should be a read operation on agiven resource $r_k$.\\

Now changing the $(op_i, op_{i+1})$ to $(op_{i+1}, op_i)$ changes the behavior of $op_i$ if and inly if $op_{i+1}$ writes $r_k$ at the file system (all the previous precedessors are still the same, only $op{i+1}$ has been added). This means by defenition the $op_{i+1}$ should be a commit operation. And, if $op_{i+1}$ is a commit operation for $T_m$, and there is at least one write operation in $T_m$ writing to $r_k$, then according to Corrolary 2, $op_i$ should see the most recent results and hence sees these changes. These changes could not have been seen in the first case $(op_i, op_{i+1})$ due to Corrolary 1 (No See in The Futture).\\ 


Rule 4 -\emph{Relocating the Position of an Operation Within the Sequence}: Given a sequence of operations $OP = \{op_1, ..., op_i, op_{i+1}, ..., op_j, op_{j+1}, ..., op_n\}$, $op_j$ can be put into the standing $i<j$ within the sequence (resulting in $OP = \{op_1, ..., op_i,op_j,op_{i+1},..., op_{j+1},...,op_n\}$) if and only if $\forall op \in \{op_{i+1}, ..., op_{j-1}\}$, $op$ and $op_j$ belong to different transactions and the pair $(op_j, op)$ or $(op,op_j)$ is not a pair subject to one of the conditions in Rule 3. The same holds true for $i>j$.\\

proof: We use induction to prove if assumptions above hold true $op_j$ can be relocated to $i = j-n$. Assume the same $OP$ as before. If $n = 1$ then since according to assumption the pair $(op_j, op_{j-1})$ is not subject to the conditions in Rule 3, these two can be easily exchanged. Now, lets assume the $op_j$ can be relocated to $j-n-1$, now we prove it for $n$.  After relocation to $j-n-1$, $op_{i+1}$ immediately precedes $op_j$, as according to assumption and Rule 3 $op_j$ and $op_{i+1}$ can exchange positions. After this exchange, $op_j$ has been relocated by $n$ and to $i$ and $op_i$ now immediately precedes $op_j$.\\

The same argument can be used for $i > j$.\\

%proof: If the sequence of operations $OP$ is a consistent one, exchanging the position of $a$ and $b$ does not violate this property as no difference in the bahvior of neither $op_i$ nor $op_{i+1}$ is made. If ($op_i, op_{i+1}) is changed to ($op_{i+1}, op_i) they would still both read the same data (if they are reads), since there is no other operation in the middle and Corrolary 2 says the read from $r_n$ reads the most commited write to $r-n$ and as no writes are reflected before the commit operation for that transaction is invoked, and none of these operations is a write, consequently exchanging their positions does not change the data read by them and hence the behavior.\\ 
Rule 3-  \emph{Committed transaction Should Precede the About to Commit Transaction}: $T_j$ "commits" at instant $t_j$ - $T_{commited}$ denoting the set of all commited transactions at $t_i$ such that $t_j > t_i$- only if a sequence preseving the behavior of the program can be found that in it all members of $T_{commited}$ precede $T_j$.\\

proof: $T_i$ denotes the last commited member of $T_{commited}$. Lets assume on the contrary $T_i$ does not precede $T_j$ and hence $T_j \rightarrow T_i$.\\

If there is not a pair of (read, write) or (write, read), sharing resource $r_n$ in the two transaction, then $T_i \rightarrow T_j$ according to Def 7 and Def 9.\\

If there are such operations, 1-At least an operation $a$ reads from $r_n$ in $OP_i$ sees the writes made by a write operation on $r_n$($b$) in $OP_j$, however since $T_j$ has commited after $T_i$ and writes are not reflected till commit instant, hence $a$ could not have read the writes made by $b$ and consequently $a$ does not precede $b$ contradicting the assumption $T_j \rightarrow T_i$ . OR 2- An operation $a$ in $T_i$ writes to resource $r_i$ and operation $b$ in $T_j$ reads resource $r_i$, according to Def 5 $a$ precedes $b$ and $b$ does not precede $a$, hence contradicting the assumption that $T_j \rightarrow T_i$.\\ 


The same reasoning could be done for $T_i$ and $T_{i-1}$, $T_{i-1}$ the last commited transaction before $T_{i}$. Now according to rule 3, since $T_{i-1}$ precedes $T_i$ and $T_i$ precedes $T_{j}$, hence $T_{i-1}$ precedes $T_{j}$. Same reasoning is applied for all the members of $T_{commited}$. \\

%If an operation $op_{i_1}$ in $OP_i$, writes some data used by an operation $op_{j_i}$ in $OP_j$, then $op_{j_1}$ can not precede $op_{i_1}$ since then it could not have seen the changes by it, however def 3 says it does, this is in contrats to the assumption. On the other hand, if $op_{j_1}$ affects some data used by $op{i_1}$, then since $op{i_1}$ has not seen the change, $op{j_1}$ can not precede $op{i_1}$. If no operations in $OP_i$ affects another in $OP_j$, then the set of all opeartions in $OP_i$ both precede and succed those in $OP_j$, and hence the contraray to the assumption. In this case, the transaction are conceptually interchangable. In either case, $T_j$ either "succeds" or "succeds and precedes" $T_i$. Either way it succeds the other. \\




Rule 3-\emph{Operation in the Committed Transactions Precede Those in The Transaction About to Commit}: $T_j$ "commits" at instant $t_j$ - $T_{commited}$ denoting the set of all commited transactions at $t_i$ such that $t_j > t_i$- only if $\forall op_i \in OP_{T_i}$, $\forall op_{all} \in$ the union of $OP_j$, such that $OP_j$ is the set of operations  for every $T_j \in T_{commited}$, $op_{all}$  should precede $op_{i}$.\\

proof: Follows immediately from Rul2 and Def 7.\\

Rule 4- \emph{If All committed Transactions Precede A Transaction About to Commit, It Commits}: $T_i$ commits if at its commit instant, $\forall T_j \in T_ccommitted T_j \rightarrow T_i$.

proof: If $\forall T_j \in T_committed T_j \rightarrow T_i$, a total ordering can be established corresponding the precedence relationship and all writes made by committed transaction have been seen by reads in $T_i$, hence according to Def 8, a consistent sequence of transactions would be provided. 

\subsection{TransactionalFile Opeartions}
TransactioalFiles can be either created inside or outside $T$. The TransactionalFile object ensures transactional access no matter where it is created and accessed. Meaning the first time $T_i$ \emph {uses} $tf_j$ the flow of the program should look like no other piece of code (i.e Transactional and non-transactional) outside $T_i$ will \emph {modify} $tf_j$ before $OP(endtransaction)_{T_i}$. Otherwise, $T_i$ has to abort. \\


A Transactionalfile can be shared among any subset of the members of $T$. If shared, the offset is shared between these too. 

\section{Implications for the System}

\subsection{General Implications}
Now, if $T_i$ and $T_j$ do not either write the data the other one reads, according to Rule 2 and 3 both can commit. This means even if $T_i$ accesses $r_k$ before $T_j$ does, but $T_j$ commits before $T_i$, given that $T_j$ has not modified the data in use by $T_i$, $T_i$ could still commit. \\


It should be noted that according to Rule 3 even if all but one of the transactions in the set of committed transaction precede $T_i$ which is about to commit, , still the transaction can not commit. Therefore it takes only one operation to make the transaction dependent (all the committed transaction should precede this transaction but can not since one of the operations in the committed transaction does not precede an operation in this transaction). Formally, if there is an operation $b \in OP_{T_j}$ and an operation $a \in OP_{T_i}$ such that $a$ does not precede $b$ then $T_i$ can not precede $T_j$. In the rules below $T_i$ is not commited yet and $T$ denotes the set of all commited or active transactions. \\

%Assumption : The reads within transaction do actually occur when they are encountered. However, no change to the file system is done before commit instant as said before (writes, changing the offset and etc.)\\


%Assumption: seek operation means after this operation is executed, the offset should be at the chosen value regardless of the actual value of the offset in the file system.\\

%Principle 0: An operation should not see any modifications other than those made by precedessors (either in the same transaction or others)\\

proof: Obviously such modifications are not valid and are probably part of a faulty operation.\\


Principle 1) \emph{Reads Should be Validated At Commit Instant}: If an operation $a$ that "reads" some data $r_n$ (reads the data at some $t < t_{commit-of-the-transaction}$(the commit instant)), it should be ensured that the data $r_n$ is still valid in the actual file system at $t_{commit-of-the-transaction}$ (commit instant) or $T_i$ has to abort.\\

proof: If the data is not valid anymore it means the data $r_n$ has been written since $t_{i-1}$. This implies at least one operation $b$ has written the data since the use and $b$ does not precede $a$ and $a$ precedes $b$. $b$ is either an operation in the same transaction or a different one. According to Def 4, however if $b$ is in the same transaction, then $b$ can not have commited yet and hence the changes are not reflected in the filesystem so it is still valid at commit instant. Hence, $b$ belongs to a different transaction namely $T_j$. Since $T_j$ should have already commited, according to Rule 3 all operation in $OP_{T_j}$ should precede those in $OP_{T_i}$, however, we know there is at least an operation $b$ that does not precede $a$ (since $a$ has not seen the "writes" made to $r_n$ by $b$). Hence, $T_i$ can not commit.\\

The same reasoning could be done for $T_i$ and $T_{i-1}$, $T_{i-1}$ the last commited transaction before $T_{i}$. Now according to rule 3, since $T_j$ succeds $T_i$ and $T_i$ succeds $T_{i-1}$, hence $T_j$ succeds $T_{i-1}$. Same reasoning is applied for all the members of $T_{commited}$. \\ 

Principle 2: emph{If Reads Are Valid At Commit Instant, the Transaction Commits}: If by applying Principle 1 for $T_i$ it is ensured that $\forall r_i \in R{T_i}$ (all data read by $T_i$) is still valid at commit instant, then $\forall T_j \in T_{committed} T_j \rightarrow T_j$ and hence according to Rule 4 $T_i$ commits.\\

proof: If all data read is still valid at commit instant, means all operation in the set of operations belonging to committed transactions, precede those in $T_i$ (since no writes have been seen), and consequently all those transactions precede $T_i$.\\ 

%Principle 2) 1-A seek operation in $OP_{T_i}$ does not make $T_i$ dependent on any data and hence does not count as "using" ("use" as defined in Def 1).\\

%The seek operations seeks to an offste within the file descriptor regardless of the changes made by opretaions in all transaction, hence the result would always be the same so according to Def 2 it does not count as using.\\ 

%rinciple 3) 1- A read operation in $OP_{T_i}$ "uses" contents read and makes $T_i$ dependent on the file contents read.\\

%proof: Due to assumption 1 and since the data read by this operation can be used for computation, in case the data read would change, the data and hence computations results would be different. According to Def 2 it uses the data\\

%Principle 4) 1- A write operation in $OP_{T_i}$ does not make $T_i$ dependent on file contents.\\   

%proof: The result of a write would always be the same contents thus, according to Def 2 does not count as using.\\

%\subsection{A Model for Offset and Data Dependeny for Operations}

%Assumption: Each operation has a offset status assioated with it. We call this operation.offsetstatus. Each operation has a data satus assosiated with it namely operation.datastatus\\

%Assumption:  An operation with Speculative offset means the offset that the operation would be perfomed at can be relocated in the commit instant. An operation with absolute offset means the value is absolute and does not depend on any other operation OR the operation only specifies a specific value for the offset for the next operation (i.e. seek). An operation with defenite offset means the offset value that operation has to be carried on from, can not be relocated, but still could have changed before commit instant by other committed transaction and not being valid at commit instant\\

%Assumption: Based on previous assumptions read.datastatus = Defenite and others are Speculative, hence they do not use the "file contents"

%Principle 5: an operation with defeinte offset status assosiated with it is the only one "using" the offset value.\\

%proof: For speculative and absolute it comes form the defeniton that they do not count as using, since one is determined at commit instant and hence read at the same time, and the other is always the same value. However, with defenite, if some other transaction changes the offset of the file descriptor, then the result of the operation would have been possibly different. According to Def 2 it uses the offset data.\\ 




%Def: We define a function $f$ that takes as input an operation (i.e. seek, read, write) and gives as the output the offsetstatus assosiated with that operation. In speaking, $f$ determines the offsetstatus assoiated with an operation given that it is the ONLY operation in the operation set of the transaction. \\

%1- $f$(Seek) = Absolute 

%proof: seek operation does not invlove reading offset value hence, according to assumptions it is considered as absoulute.\\

%2-$f$(Read) = Defenite

%proof: Since according to Principle 3, read is actually done before the commit instant and hence can not be relocated in the commit instant. According to Principle 1 and 5 the offset the read is done from should not be affected by other members of $T$ before this transaction gets to its commit instant. This means not the only offset can not be relocated, but a check should be done at the commit instant for its validity.\\

%3-$f$(Write) = Speculative

%proof: If write is the only operation, then since according to Principle 4 the write does not use any file contents before commit instant, hence the ofsset value can be determined at commit instant from the most recently commited offset value by then. Then the write is done on the offset. This preserves the semantics since the commit instant is atomic by defenition and Rule 6 and 5 are preserved, as the write operation is getting the most recent data(offset) and hence would see all the changes made to it by commited transactions so far and hence this write operation would succed all the operations in other commited transaction. Consequently this transaction can succed those.\\   

%Requirements: To determine an operation should use Absolute, Define or Speculative offset type (the first and third would be used in commit instant) respective to other operatins in the set, the requirements below should be provided:\\

%Req 1- 1-Any operation follwoing an operation $op$ (which belongs to the same set of operations) with offsetstatus = Absolute, has offsetstatus = Absolute itself. 2-Any operation that applying $f$ to it would yield in Absolute, has offsettype = absolute \\

%proof: 1-Lets assume $op$ is the first operation with offsettype = Absolute in the set of operatopns. The immediate operation ($op_{imm}$) after $op$ uses the absolute offset specified by $op$ or the absolute offset advanced by a certain number in case of read or write. $op_{imm}$ either advances this value or specifies another absolute value, in either case the offset type for $op_{imm}$ would be absolute. The same reasoning could be done for all the follwing operations.\\

%2- Since such an operation is seek and seek regardless of all other operation (in and out of transaction) sets the offset and hence is always absolute.\\

%Req 2- 1- Any operations $op$ that applying $f$ on it would yield in Definite, has offsettype = Defenite unless speified otherwise by Req 1  2-Any operation immediately follwoing an operation $op$ (which belongs to the same set of operations) with offsetstatus = Defenite, has offsetstatus = Defenite itself. unless specified otherwise by Req 1\\

%proof: 1- if $f$ makes the output Defenite it means this operation should "use" offset value at that time unless some previous instruction has set the offset to specific value independent of other transactions and thus changing the offset value would not have any effect on the outcome of $op$ as the value $op$ should be carried on from would be independent. Operation having such an effect is the one with Absolute offsettype.\\

%2- The following operation should be Defenite type since $op$ has bound the offset to a value that should be validated at commit instant (Principle 1), unless it is a seek that only sets the offset.\\ 

%Req 3: Any operation with offsetstaus = Defenite, should change the offsetstatus for prveious operations with a value of Speculative, to Defenite.\\

%If $op.offsettype = Definte$ means no previous operations are Absolute (Req 1). So it just could be a mixture of Speculative and Defenite. Lets assume $op_{Def}$ as the first operation after a $op_{spec}$ a Speculative one (Rq 2 ensure after A Defenite there is no Speculative).  $op_{Def}$ makes the offset bound to a specific value and $op_{Def} succeds $op{Spec}$ so it should be able to see the changes made by $op{Spec} this neccesitaes that $op{Spec}$ be done at a specific offset such that when advanced as the effect of operation, it gives the offset chosen by $op{Def}$, otherwise $op{Def}$ does not see the effects of $op{Spec}$ which is in contradiction to presumtions. Hence, $op_{Spec}.offsettype$ should be converted to Defenite.\\

%Req 4: Any operation $op$ which applying $f$ on it yields in Speculative has offsettype=Speculative unless specified otherwise by previous requirements.\\

%proof: If we denote $OP_{prev}.offsettype$ as the offsettypes for the prviously encountered operations, considering all requirements this means all (if any) the members of $OP_{prev}$ should be Speculative for $op$ to be able to be Speculative. In this case the offset passed to $op$ is Speculative and if $op$ is Speculative as considered alone then it would carry on the actions in the passed-in Speculative offset.\\ 

%Req 5: An operation with datastatus = Defenite, converts all the speculative offsttypes of the previous operations to Defenite. This requirement is always enforced.\\

%proof: Since if datastatus = Defenite, this mean the operation is "using" or "reading" some file contents- an irreversible action- and as this operation should be able to see the modifications made by all preceding operations, if these are speculative, at commit instant they may "relocate" their modifications, this means eitehr $op$ may not see some changes made by precedessors or may see some changes not made by any precedessors (since the speculatively changes were used by $op$ however these did not get reflected in the actual file system since they were relocated at commit instant). Either way contradicts the defenitions and principles (Def 8 and Principle 0).\\

%Note: Req 5 suggest if datatype = Definite this causes all the previous operation with Speculative as offsettype to be changed to Defenite.

%Def: We define a function $g(offsetstat_1, offsettat_2, ..., offsetstat_{n-1}, operation)$ that takes an input  $n-1$ number of $op.offsetstatus$ plus an operation and gives as the output $n$ number of $op.offsetstatus$. The $g$ functions provides the requirements mentioned above. Follwing rules determine the output, the rules are ordererd according to their priority:\\

%1- if $f(operation) = Absolute$, $g(offsetstat_1, ..., {offsetstat_{n-1}}, operation) = ({offsetstat_1}, ..., {offsetstat_{n-1}}, Absolute)$ leaving all offsetstats intact.\\

%2- if $f(operation) = Defenite$:\\

%\hspace{ 8mm}
%	2.1- if $\exists offset \in \{offsetstat_1, ..., offsetstat_{n-1}$\} such that offset = Absolute,  $g(offsetstat_1, ..., offsetstat_{n-1}, operation) = (offsetstat_1, ... ,Absolute)$ leaving all offsestats intact.\\ 

%\hspace{ 8mm}
%	2.2- else $\forall offsetstat \in (offsetstat_0, ..., offsetstat_{n-1})$ if offsetstat = Speculative, change its value to Defenite in the output of $g$, plus appending a Defenite to the output (for instane $g(speculative, speculative, read) = (defenite, defeinte, definte)$)\\

%3- if $f(operation) = Speculative$:\\

%\hspace{ 8mm}
%	3.1- if $\exists offset \in \{offsetstat_1, ..., offsetstat_{n-1}\}$ such that offset = Absolute, $g(offsetstat_1, ... offsetstat_{n-1}, operation) = (offsetstat_1, ... Absolute)$ leaving all offsestats intact.\\ 

%\hspace{ 8mm}
%	3.2- else if $\exists offsetstat \in \{offsetstat_1, ..., offsetstat_{n-1}\}$ such that offsetstat = {Defenite},  $g(offsetstat_1, ...,  offsetstat_{n-1}, operation) = (offsetstat_1 ,..., offsetstat_{n-1}, Defenite)$\\

%\hspace{ 8mm}
%	3.3- else $g(offsetstat_1, ..., offsetstat_{n-1}, operation) = (offsetstat_1, ..., offsetstat_{n-1}, Speculative)$\\

%4- After doing the above check if datastatus = Defenite, then change all offsettypes in the input that are Speculative to Defenite in the output.

%51, 2.1, 3.1 ensure Req 1. 2.2 and 3.2 ensures Req 2. 2. ensure Req 3, and finally 3.3 provides Req 4. 4 also ensures Req 5. 








%$f: \{op_0.offsetstatus,op_1.offsetstatus,...,op_{n-1}.offsetstatus\} \rightarrow \{op_0.offsetstatus, op_1.offsettatus,...,op_{n-1}.offsetstatus,op_m.offsetstatus, \}



%2- Since due to assumption 1, the read should be actually done, it has to be done from a real offset, and since the commited offset at the commit instant is unknown in adavance, the currenlty most recent commited offset is chosen. This makes the read and hence transaction dependent on the offset value read.

%2- Since all the following read and writes at least succed this seek operation, they should be able to see the changes made by it unless some other operation in between changes the data in question (offset). Lets first assume the operation ($op$) following the seek. If $op$ is a read or write, then it would carry on reading or writing the data from the absolute offset so it would advance the aboslute offset resulting in another absolute offset and hence no "use" of the actual offset would be applicable. If the following is a seek, then the resultant offset would be another absolute offste. Same reasoning could be done for all the succeding operations and hence the offset would be still absolute. To prove only seek can make the offset absolute it should have been made absolute by read or write, however as we'll see in Principles 2 and 3 they can not yield in an absolute offset. Hence, an absolute offset indicates there has been a seek.\\
%According to priciple 1 the offset chosen should be validated at the the commit instant.\\
\section{Implementation Overview}

%proof: A write operation writes to the most recently commited offset (unless imposed otherwise by a seek or a prior read or a following read as shown above) of the file descriptor.  According to Def 2, it does not "use" any data. Since the write does not happen till commit instant, the offset chosen during the execution of transaction can be speculative, and the actual offset to be written to, is determined at the sommit instant, when the most recently commited offset is known. 
Now we will describe the components of our system and demosntrate how the algorithm conforms to the function just described, and how all the rules principles are gurantedd. 









to be continued

%\begin{tabular}{c|c|c|}
\end{document}
